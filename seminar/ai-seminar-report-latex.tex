\documentclass[12pt]{article}
\usepackage{geometry}
\usepackage{titlesec}
\usepackage{graphicx}
\usepackage{hyperref}

\geometry{a4paper, margin=1in}
\titleformat{\section}{\large\bfseries}{\thesection}{1em}{}
\titleformat{\subsection}{\normalsize\bfseries}{\thesubsection}{1em}{}

\title{Seminar Report: Algorithms for Planning as a State Space Search}
\author{Likhith K.}
\date{}

\begin{document}

\maketitle

\section{Overview}
The seminar focused on AI planning algorithms, exploring state space search techniques in artificial intelligence. The presentation provided a comprehensive examination of planning approaches, with a particular emphasis on forward and backward search methods and the Fast Forward (FF) planning system.

\section{Key Concepts}

\subsection{Introduction to AI Planning}
AI planning is a fundamental problem-solving approach in artificial intelligence that involves:
\begin{itemize}
    \item Deciding on a sequence of actions in advance
    \item Transforming an initial state into a goal state
    \item Enabling autonomous systems to make complex decisions
\end{itemize}

\subsection{Importance of AI Planning}
The presentation highlighted the critical role of planning in:
\begin{itemize}
    \item Autonomous systems
    \item Decision-making processes
    \item Applications across various domains including:
    \begin{itemize}
        \item Robotics
        \item Logistics
        \item Game AI
    \end{itemize}
\end{itemize}

\subsection{State Space Representation}
The core components of AI planning were identified as:
\begin{itemize}
    \item Initial State: The starting point
    \item States: Descriptions of the world at a particular time
    \item Actions: Operations that can change the state
    \item Goal State: The desired final configuration
\end{itemize}

\section{Search Approaches}

\subsection{Forward Search}
Characteristics:
\begin{itemize}
    \item Starts from the initial state
    \item Expands the graph by computing successors
    \item Computes resulting states using applicable actions
    \item Challenges include dealing with a large number of irrelevant actions
\end{itemize}

\subsection{Backward Search}
Key features:
\begin{itemize}
    \item Begins with the goal state
    \item Expands the graph by computing parents
    \item Uses regression to determine previous state conditions
    \item Reduces issues with irrelevant actions
\end{itemize}

\section{Fast Forward (FF) Planning System}

\subsection{Unique Approach}
The FF system introduces innovative planning strategies:
\begin{itemize}
    \item Heuristic method using relaxed planning graph
    \item Enforced hill climbing search method
    \item Intelligent action ordering
\end{itemize}

\subsection{Relaxed Planning Graph}
Distinctive characteristics:
\begin{itemize}
    \item Removes deleted lists of actions
    \item Expands graph until all goals are contained
    \item Eliminates mutexes for simplified planning
\end{itemize}

\subsection{Heuristic Computation}
\begin{itemize}
    \item Estimates solution length
    \item Computed in polynomial time
    \item Admissible heuristic based on positive state facts
\end{itemize}

\subsection{Enforced Hill Climbing}
Unique features:
\begin{itemize}
    \item Evaluates all successors
    \item Performs breadth-first search when no better heuristic is found
    \item Effectively navigates plateaus and local minima
\end{itemize}

\section{Performance and Significance}
\begin{itemize}
    \item Demonstrated success in the AIPS-2000 planning competition
    \item Provides an innovative approach to automated reasoning
    \item Showcases how intelligent heuristics can optimize planning processes
\end{itemize}

\section{Conclusion}
The seminar presented a deep dive into AI planning algorithms, emphasizing the complexity and sophistication of state space search techniques. The Fast Forward system, in particular, was highlighted as a significant advancement in automated planning research.

\section{Recommendations for Further Study}
\begin{itemize}
    \item Explore implementation details of the FF planning system
    \item Investigate applications in more complex real-world scenarios
    \item Compare FF with other planning algorithms
\end{itemize}

\end{document}
